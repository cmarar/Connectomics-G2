\documentclass{article}
\usepackage[utf8]{inputenc}
\usepackage{graphicx}

\title{Constructing a Connectome for Alzheimer's Treatment}
\author{Ajay Ananthakrishnan, Elizabeth Daramola, Kevin Drucia, Carolyn Marar }
\date{January 2020}

\begin{document}

\maketitle

\section{Scientific Question}
Alzheimer's disease (AD) is an untreatable neurodegenerative disorder which impairs a person's ability to think, recall memories, and perform basic tasks. Around 5.5 million Americans are estimated to suffer from AD, suggesting that this disease may be the third leading cause of death in the elderly. \cite{Alzheimer's Facts} While treatments are being developed for other neurodegenerative diseases with limited success, there are currently no treatments for AD. As the population ages, AD is becoming an urgent public health issue in need of a solution.

Deep Brain Stimulation (DBS) is a therapeutic technique in which electrodes are implanted into the brain. Electrical signals are used to stimulate the surrounding neurons and regulate abnormal behavior. DBS has been approved for a number of ailments \cite{DBS uses}, and it has been shown to significantly improve symptoms of Parkinson's disease. \cite{DBS PD} Little research has been done on the effects of DBS in AD. One study saw no significant improvement in cognitive tests of AD patients after DBS of the fornix. \cite{DBS AD} Much more research must be conducted before ruling DBS therapy out for AD treatment, though. DBS may prove more effective in a different region of the brain. Our study aims to determine if this is the case.

We propose to use connectomes to effectively target areas of the brain with DBS as a means to treat AD patients. We will first generate connectomes for AD patients at different stages of their disease progression to identify regions with the greatest degeneration. These connectomes will be used to simulate DBS in several regions to determine the best location for electrode implantation. Over the next year, we will perform cognitive tests on the patients to determine whether DBS treatment translates to cognitive and behavioral improvement. These results will be compared against the simulations in order to assess its accuracy. We hope to not only understand the means in which different brain regions ‘break down’ in terms of connectivity, but also find whether DBS may be therapeutic in AD and if connectomes can be used to predict this outcome. 


\section{Hardware and Facilities}
\section{Data Collection}
\section{Data Processing and Information Extraction}
\section{Data Storage}
\section{Estimated Cost}



\begin{thebibliography}{9}

\bibitem{Alzheimer's Facts} 
National Institute on Aging. 
\textit{Alzheimer's Disease Fact Sheet}. 
National Institute on Aging, National Institutes of Health, Bethesda, Maryland, 2019.
\\\texttt{https://www.nia.nih.gov/health/alzheimers-disease-fact-sheet}

\bibitem{DBS uses} 
Mayo Clinic Staff. 
\textit{Deep Brain Stimulation}. 
Mayo Clinic, 2020.
\\\texttt{https://www.mayoclinic.org/tests-procedures/deep-brain-stimulation/about/pac-20384562}

\bibitem{DBS PD} 
Simon Little, Martijn Beudel, Ludvic Zrinzo, Thomas Foltynie, Patricia Limousin, Marwan Hariz, Spencer Neal, Binith Cheeran, Hayriye Cagnan, James Gratwicke, Tipu Z Aziz, Alex Pogosyan, Peter Brown. 
\textit{Bilateral adaptive deep brain stimulation is effectivein Parkinson’s disease}. 
J Neurol Neurosurg Psychiatry, 2016; 87:717–721.

\bibitem{DBS AD} 
Lozano, A. M., Fosdick, L., Chakravarty, M. M., Leoutsakos, J. M., Munro, C., Oh, E., … Smith, G. S. 
\textit{A Phase II Study of Fornix Deep Brain Stimulation in Mild Alzheimer’s Disease}. 
Journal of Alzheimer's disease, 54(2), 777–787. 

\end{thebibliography}

\end{document}
