\documentclass{article}
\usepackage[utf8]{inputenc}
\usepackage{graphicx}

\title{Constructing a Connectome for Alzheimer's Treatment}
\author{Ajay Ananthakrishnan, Elizabeth Daramola, Kevin Drucia, Carolyn Marar }
\date{January 2020}

\begin{document}

\maketitle

\section{Scientific Question}
Alzheimer's disease (AD) is an untreatable neurodegenerative disorder which impairs a person's ability to think, recall memories, and perform basic tasks. Around 5.5 million Americans are estimated to suffer from AD, suggesting that this disease may be the third leading cause of death in the elderly.\cite{Alzheimer's Facts} While treatments are being developed for other neurodegenerative diseases with limited success, there are currently no treatments for AD. As the population ages, AD is becoming an urgent public health issue in need of a solution.

Deep Brain Stimulation (DBS) is a therapeutic technique in which electrodes are implanted into the brain. Electrical signals are used to stimulate the surrounding neurons and regulate abnormal behavior. DBS has been approved for a number of ailments\cite{DBS uses}, and it has been shown to significantly improve symptoms of Parkinson's disease.\cite{DBS PD} Little research has been done on the effects of DBS in AD. 


\section{Hardware and Facilities}
\section{Data Collection}
\section{Data Processing and Information Extraction}
\section{Data Storage}
\section{Estimated Cost}



\begin{thebibliography}{9}

\bibitem{Alzheimer's Facts} 
National Institute on Aging. 
\textit{Alzheimer's Disease Fact Sheet}. 
Mational Institute on Aging, National Institutes of Health, Bethesda, Maryland, 2019.
\\\texttt{https://www.nia.nih.gov/health/alzheimers-disease-fact-sheet}

\end{thebibliography}

\end{document}
